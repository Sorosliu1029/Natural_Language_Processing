\documentclass[a4paper, twocolumn]{article}
\usepackage{CJKutf8}
\usepackage[sc]{mathpazo} % Use the Palatino font
\usepackage[T1]{fontenc} % Use 8-bit encoding that has 256 glyphs
\linespread{1.2} % Line spacing - Palatino needs more space between lines
\usepackage{microtype} % Slightly tweak font spacing for aesthetics

\usepackage[english]{babel} % Language hyphenation and typographical rules

\usepackage[hmarginratio=1:1,top=32mm,left=20mm,right=20mm,columnsep=20pt]{geometry} % Document margins
\usepackage[hang, small,labelfont=bf,up,textfont=it,up]{caption} % Custom captions under/above floats in tables or figures
%\usepackage{booktabs} % Horizontal rules in tables

\usepackage{lettrine} % The lettrine is the first enlarged letter at the beginning of the text

\usepackage{enumitem} % Customized lists
\setlist[itemize]{noitemsep} % Make itemize lists more compact

\usepackage{abstract} % Allows abstract customization
\renewcommand{\abstractnamefont}{\normalfont\itshape\bfseries} % Set the "Abstract" text to bold
\renewcommand{\abstracttextfont}{\normalfont} % Set the abstract itself to small italic text

\usepackage{titlesec} % Allows customization of titles
\renewcommand\thesection{\Roman{section}} % Roman numerals for the sections
\renewcommand\thesubsection{\roman{subsection}} % roman numerals for subsections
\titleformat{\section}[block]{\large\scshape\centering}{\thesection.}{1em}{} % Change the look of the section titles
\titleformat{\subsection}[block]{\large}{\thesubsection.}{1em}{} % Change the look of the section titles

\usepackage{fancyhdr} % Headers and footers
\pagestyle{fancy} % All pages have headers and footers
\fancyhead{} % Blank out the default header
\fancyfoot{} % Blank out the default footer
\fancyhead[C]{
	\begin{CJK}{UTF8}{gbsn}
	《中文信息处理》课程项目报告
	\end{CJK}}
\fancyfoot[RO,LE]{\thepage} % Custom footer text
\usepackage{titling} % Customizing the title section

\usepackage{hyperref} % For hyperlinks in the PDF
\hypersetup{hidelinks}

\usepackage{graphicx}
\renewcommand{\normalsize}{\fontsize{10.5pt}{\baselineskip}\selectfont}
%----------------------------------------------------------------------------------------
%	TITLE SECTION
%----------------------------------------------------------------------------------------

\setlength{\droptitle}{-4\baselineskip} % Move the title up

\pretitle{\begin{center}\Huge\bfseries} % Article title formatting
	\posttitle{\end{center}} % Article title closing formatting
\title{基于词包和词向量的电影评论情感分析} % Article title
\author{%
	\textsc{刘阳 13307130167} \\[1ex] % Your name
	\normalsize 复旦大学 计算机学院 \\ % Your institution
}
\date{} % Leave empty to omit a date

%----------------------------------------------------------------------------------------

\begin{document}
\begin{CJK}{UTF8}{gbsn}
	% Print the title
	\maketitle
	
	%----------------------------------------------------------------------------------------
%		ARTICLE CONTENTS
	%----------------------------------------------------------------------------------------
	\section{摘要}

	\section{关键词}

	\section{引言}

	\section{数据清洗}
使用的语料数据来自Kaggle.com,其中有25,000的电影评论打了标签,另有75,000的评论没有打标签。\\
首先查看原始语料,任取一条评论数据,字数为433, 其中的片段如下:
``<br /><br />The actual feature film bit when it finally starts is only on for 20 minutes or so excluding the Smooth Criminal sequence and Joe Pesci is convincing as a psychopathic all powerful drug lord.'' \\
可以看到原始文本有网页的标签和具体的数字。
网页标签应该是在爬取电影评论网站时没有完全解析导致的遗留。
而数字因为本身不能单独记为一个词,所以也需要处理掉。
同时对于特征提取没有多大帮助,而又会增加分类任务空间和时间复杂度的停用词,我们也要去掉。\\
对于网页标签,我们使用BeautifulSoup这个网页解析库来提取去除网页标签的文本。
而对于数字,我们利用正则表达式,来匹配数字,替换成空格。
对于标点符号,我们也是先替换成空格。
在之后有对段落分句的需求时,再使用原始语料库进行分句。
对于停用词,我们使用NLTK提供的停用词语料库,来查找和删除评论中的停用词。
最后的清洗结果还需要对所有字母转成小写。
对于上面考察的那条评论数据,在经过这一系列的处理后,字数为219。
	%----------------------------------------------------------------------------------------
	%	REFERENCE LIST
	%----------------------------------------------------------------------------------------
	\renewcommand{\refname}{参考文献}
	\begin{thebibliography}{99} % Bibliography - this is intentionally simple in this template
		
		\bibitem{1}
		
		\newblock {\em} 
		\newblock 
		
\end{thebibliography}
	
	%----------------------------------------------------------------------------------------
\end{CJK}	
\end{document}
